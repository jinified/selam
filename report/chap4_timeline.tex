\chapter{Plan for the next semester}
\section{Implement more robust object detection and object tracking algorithms}

\subsection{Improve object detection}
Moving on, achieving consistent and accurate detection of objects in more difficult environments such as partial occlusion, sudden illumination change and presence of shadow. Taking inspirations from success of top object detector such as \outcite{Yang2014} that aggregate multiple-channel features; a multi-cues approach that includes both global and local features will be implemented to increase robustness of current object detector.

\subsection{Better object tracking}
One limitation of current tracking approach is neglecting prior tracked position which can help to prevent drift and reduce false positives. Instead of traditional particle filter that ignores current measurement in its system model and use only color model for its observation model, the project look to integrate more features such as optical flow and salient features. 

\section{Automate parameters selection and model selection}
Unlike approach taken by other schools that rely on manual visual tuning to achieve robust object tracking, the project aims to take inspirations from works of \cite{Zhang2016} and \cite{collins2005online} that attempt to map algorithms-parameters pair to specific dataset. A similarity function is then used to measure difference between test images with trained images.

\section{Experimental Setup}
In order to perform evaluation on proposed vision framework, several performance metrics will be used but the project adopts approach of \outcite{Luo2014} that uses: a) MOTA (Multiple Object Tracking Accuracy) b) MOTP (Multiple Object Tracking Precision) Average Overlap (Intersection-over-Union of bounding box). 

\subsection{Real-world dataset}
Recorded images from Robosub 2015, Robosub 2016 and Queenstown Pooltest will be labelled and used to evaluate performance of proposed vision algorithms. These datasets will be divided according to different challenges such as buoy detection, bins detection and coins detection.

\section{Proposed Time-line}
\begin{center}
    \begin{tabular}{ | l | p{10cm} |}
    \hline
    Time Period & Work to be done \\ \hline
    December Holiday & Propose and implement a set of object detection and object tracking that will work in different underwater conditions
    \\ \hline
    January & Validate proposed approach by comparing their performance with the baseline.
    Identify ways to automate parameter selection and model selection based for different tasks or scenarios
     \\ \hline
    February, March, April & Report writing on experimental results and findings
    \\
    \hline
    \end{tabular}
\end{center}